\documentclass[aspectratio=169]{beamer}
\input{preamble.tex}

\usepackage{fontawesome}

% То, что в квадратных скобках, отображается внизу по центру каждого слайда. 
\title[Тестирование библиотек на IMG]{Исследование производительности умножения матриц на графических ускорителях от Imagination Technologies}

% То, что в квадратных скобках, отображается в левом нижнем углу. 
\institute[СПбГУ]{}

% То, что в квадратных скобках, отображается в левом нижнем углу.
\author[Пучкин Станислав]{Пучкин Станислав Андреевич, группа 23.Б08-мм}
 
\begin{document}
{
\setbeamertemplate{footline}{}
% Лого университета или организации, отображается в шапке титульного листа
\begin{frame}
  \includegraphics[width=1.4cm]{pictures/SPbGU_Logo.png}
\vspace{-35pt}
\hspace{-10pt}
\begin{center}
   \begin{tabular}{c}
        \scriptsize{Санкт-Петербургский государственный университет} \\
        \scriptsize{Кафедра системного программирования}
    \end{tabular}
\titlepage
\end{center}

\btVFill

{\scriptsize
  % У научного руководителя должна быть указана научная степень
   \textbf{Научный руководитель:} С. В. Григорьев, доцент кафедры системного программирования \\
 }
\begin{center}
  \vspace{5pt}
  \scriptsize{Санкт-Петербург\\
                 2025}
  \end{center}

\end{frame}
}

\begin{frame}
    \frametitle{Введение}
    \begin{itemize}
    \item GEMM — ключевая операция линейной алгебры
    \item Используется в ML, CV, численных методах
    \item Производительность зависит от GPU
    \item OpenCL — кроссплатформенный стандарт
    \item Актуальность для архитектуры RISC-V
  \end{itemize}
\end{frame}

% Обязательный слайд: четкая формулировка цели данной работы и постановка задачи
% Описание выносимых на защиту результатов, процесса или особенностей их достижения и т.д.
\begin{frame}
  \frametitle{Постановка задачи}
  \textbf{Целью} работы является исследование производительности различных реализаций умножения матриц из библиотек MyGEMM и CLBlast на одноплатных компьютерах с архитектурой RISC-V с графическим ускорителем от Imagination Technologies\\   %озвученной выше  
  \textbf{Задачи}:
  \begin{itemize}
    \item Обновление системных пакетов и программного обеспечения
    \item Провести серию тестов производительности различных ядер библиотеки MyGEMM
    \item Выполнить процедуру автоматического тюнинга библиотеки CLBlast и провести измерения производительности.
    \item Провести сравнительный анализ производительности MyGEMM и CLBlast
  \end{itemize}
\end{frame}

\begin{frame}
  \frametitle{Контекст исследования}
  \begin{itemize}
      \item GEMM
      \item CUDA vs OpenCL
      \item RISC-V и Imagination Technologies
      \item MyGEMM и CLBlast
  \end{itemize}
\end{frame}

\begin{frame}
    \frametitle{MyGEMM и CLBlast}
    \begin{table}[]
    \centering
    \begin{tabular}{p{0.45\textwidth}p{0.45\textwidth}}
    \hline
    \textbf{MyGEMM} & \textbf{CLBlast} \\
    \hline
    Учебная библиотека, демонстрирующая пошаговую оптимизацию & Отпимизированная библиотека для использования в производительных решениях \\
    \hline
    11 отдельных ядер с нарастающей сложностью оптимизаций & Единое универсальное ядро с адаптивными параметрами \\
    \hline
    Ручная настройка параметров для каждой платформы & Автоматический поиск оптимальных параметров \\
    \hline
    Фокус на понимании принципов оптимизации & Фокус на максимальной производительности \\
    \hline
    \end{tabular}
    \end{table}
\end{frame}

\begin{frame}
  \frametitle{Условия эксперимента}
  \begin{itemize}
    \item \textbf{Платформы:}
      \begin{itemize}
        \item Banana Pi BPI-F3, GPU IMG BXE-2-32
        \item StarFive VisionFive 2, GPU IMG BXE-4-32 MC1
        \item Intel Core i9-12900H, GPU Intel Iris Xe Graphics
      \end{itemize}
    \item \textbf{Условия тестирования:}
      \begin{itemize}
        \item Форк репозитория (\url{https://github.com/stanislavgexrby/myGEMM/tree/develop}) со скриптом \texttt{run\_all\_kernels.sh} для MyGEMM
        \item Репозиторий (\url{https://github.com/vkutuev/matrix-benchmark/tree/main}) со скриптами \texttt{benchmark.sh} и \texttt{conf.sh} для CLBlast
      \end{itemize}
    \item \textbf{Параметры тестирования:}
      \begin{itemize}
        \item Наборы параметров TS, TSM, TSN в диапазоне 8--32, 32--128 для MyGEMM
        \item Атоматический подбор параметров и размер матриц из набора 512--7680 для CLBlast
      \end{itemize}
    \item \textbf{Измеряемые метрики:}
      \begin{itemize}
        \item Время исполнения каждого ядра без учета задержек загрузки
      \end{itemize}
  \end{itemize}  
\end{frame}


\begin{frame}
    \frametitle{Результаты MyGEMM}
    \centering
    \includegraphics[scale=0.3]{pictures/comparison_best_results.png}
\end{frame}

\begin{frame}
    \frametitle{Результаты CLBlast}
    \centering
    \includegraphics[scale=0.3]{pictures/clblast_time_comparison.png}
\end{frame}

\begin{frame}
    \frametitle{Сранение MyGEMM и CLBlast}
    \centering
    \includegraphics[scale=0.45]{pictures/mygemm_vs_clblast.png}
\end{frame}

\begin{frame}
  \frametitle{Результаты}
  \begin{itemize}
    \item Библиотека MyGEMM адаптирована для работы на платформах Imagination Technologies с учётом аппаратного ограничения размера рабочей группы в 512 потоков
    \item Проведено тестирование с 100 прогонами для каждой конфигурации на матрицах $1024 \times 1024$ элементов для библиотеки MyGEMM, результаты проанализированы
    \item Для библиотеки CLBlast выполнен автоматический тюнинг с анализом результатов до и после на тестируемых платформах
    \item Проведен сравнительный анализ этих двух библиотек
  \end{itemize}
\end{frame}
%\addtocounter{framenumber}{1}
\end{document}