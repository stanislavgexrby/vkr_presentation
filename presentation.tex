\documentclass[aspectratio=169]{beamer}
\input{preamble.tex}

\usepackage{fontawesome}

% То, что в квадратных скобках, отображается внизу по центру каждого слайда. 
\title[Портирование SynGT]{Инструментальная система эквивалентных преобразований синтаксических диаграмм}

% То, что в квадратных скобках, отображается в левом нижнем углу. 
\institute[СПбГУ]{}

% То, что в квадратных скобках, отображается в левом нижнем углу.
\author[Пучкин Станислав]{Пучкин Станислав Андреевич, группа 23.Б08-мм}
 
\begin{document}
{
\setbeamertemplate{footline}{}
% Лого университета или организации, отображается в шапке титульного листа
\begin{frame}
  \includegraphics[width=1.4cm]{pictures/SPbGU_Logo.png}
\vspace{-35pt}
\hspace{-10pt}
\begin{center}
   \begin{tabular}{c}
        \scriptsize{Санкт-Петербургский государственный университет} \\
        \scriptsize{Кафедра информатики}
    \end{tabular}
\titlepage
\end{center}

\btVFill

{\scriptsize
  % У научного руководителя должна быть указана научная степень
   \textbf{Научный руководитель:} Л. Н. Федорченко, к. ф.-м. н., доцент кафедры информатики \\
 }
\begin{center}
  \vspace{5pt}
  \scriptsize{Санкт-Петербург\\
                 2025}
  \end{center}

\end{frame}
}

\begin{frame}
    \frametitle{Введение}
    \begin{itemize}
    \item Формальные грамматики — основа описания синтаксиса языков программирования
    \item RBNF (Regular Backus-Naur Form) — расширенная форма БНФ с регулярными операторами
    \item Синтаксические диаграммы (Railroad Diagrams) делают грамматики понятными человеку
    \item Оригинальная система SynGT реализована на Pascal в конце 1990-х годов
    \item Устаревшая технологическая база ограничивает использование и развитие системы
  \end{itemize}
\end{frame}

% Обязательный слайд: четкая формулировка цели данной работы и постановка задачи
% Описание выносимых на защиту результатов, процесса или особенностей их достижения и т.д.
\begin{frame}
  \frametitle{Постановка задачи}
  \textbf{Целью} работы является портирование инструментальной системы SynGT с языка Pascal на современный язык C++ с сохранением функциональности\\   
  
  \textbf{Задачи}:
  \begin{itemize}
    \item Провести анализ исходного кода системы SynGT на Pascal
    \item Разработать архитектуру портированной системы на C++
    \item Реализовать библиотеку для разбора RBNF-грамматик и построения граф-схем
    \item Разработать набор тестов для верификации корректности
    \item Создать консольное и графическое приложения
  \end{itemize}
\end{frame}

\begin{frame}
  \frametitle{Система SynGT}
  \textbf{Исходная система (Pascal):}
  \begin{itemize}
      \item Разработана на Object Pascal (Borland Delphi)
      \item ~12000 строк кода, 21 модуль компиляции
      \item Функции: разбор RBNF, визуализация, преобразования грамматик
      \item Алгоритм размещения с 7 специализированными методами
      \item Зависимость от VCL (Visual Component Library)
  \end{itemize}
  
  \vspace{0.5cm}
  
  \textbf{Проблемы:}
  \begin{itemize}
      \item Ограниченная экосистема Pascal
      \item Windows-only
      \item Отсутствие модульности и тестов
  \end{itemize}
\end{frame}

\begin{frame}
    \frametitle{Формат RBNF-грамматик}
    \textbf{Основные элементы:}
    \begin{itemize}
        \item Терминалы: \texttt{'if'}, \texttt{'begin'}, \texttt{'+'}
        \item Нетерминалы: \texttt{stmt}, \texttt{expr}, \texttt{ID}
        \item Макросы: \texttt{@list}
    \end{itemize}
    
    \textbf{Операторы:}
    \begin{itemize}
        \item Последовательность (,): \texttt{A, B, C}
        \item Альтернатива (;): \texttt{A ; B ; C}
        \item Итерация (*): \texttt{@*(A)}
        \item Группировка: \texttt{(A ; B), C}
    \end{itemize}
    
    \textbf{Пример:}
    
    \texttt{expr : term , @*(( '+' ; '-' ) , term).}
\end{frame}

\begin{frame}
  \frametitle{Архитектура портированной системы}
  \textbf{Технологии:}
  \begin{itemize}
    \item Язык: C++17 (ISO/IEC 14882:2017)
    \item Система сборки: CMake 3.20+
    \item GUI Framework: Dear ImGui с DirectX 11 backend
    \item Тестирование: Google Test 1.14.0
  \end{itemize}
  
  \vspace{0.3cm}
  
  \textbf{Компоненты:}
  \begin{itemize}
    \item \textbf{libsyngt} — основная библиотека (независимая от GUI)
    \item \textbf{syngt\_cli} — консольное приложение
    \item \textbf{syngt\_gui} — графическое приложение
    \item \textbf{tests} — модульные тесты (18 тестовых модулей)
  \end{itemize}
\end{frame}

\begin{frame}
    \frametitle{Структура библиотеки libsyngt}
    \textbf{Модули:}
    \begin{itemize}
        \item \textbf{Core} — базовые классы (Grammar, TerminalList, NonTerminalList)
        \item \textbf{Regex} — иерархия классов регулярных выражений (RETree, REAnd, REOr, REIteration)
        \item \textbf{Parser} — рекурсивный спуск для разбора RBNF
        \item \textbf{Graphics} — классы для графических объектов диаграмм
        \item \textbf{Transform} — преобразования грамматик (устранение левой рекурсии, факторизация)
        \item \textbf{Analysis} — анализ грамматик (LL(1), FIRST/FOLLOW)
    \end{itemize}
    
    \vspace{0.3cm}
    
    \textbf{Итого:} 70 файлов (~7700 строк кода)
\end{frame}

\begin{frame}
    \frametitle{Алгоритм визуализации}
    \textbf{Упрощение по сравнению с оригиналом:}
    \begin{itemize}
        \item Оригинал: 7 взаимосвязанных методов размещения
        \item Портирование: единый метод \texttt{drawObjectsToRight} для каждого типа узла
    \end{itemize}
    
    \vspace{0.3cm}
    
    \textbf{Подход:}
    \begin{itemize}
        \item \textbf{AND} (последовательность): горизонтальное размещение
        \item \textbf{OR} (альтернатива): вертикальное разветвление с точками схождения
        \item \textbf{Iteration}: цикл с обратной стрелкой
    \end{itemize}
    
    \vspace{0.3cm}
    
    \textbf{Преимущества:}
    \begin{itemize}
        \item Независимость от GUI-фреймворка
        \item Корректная визуализация всех базовых конструкций RBNF
    \end{itemize}
\end{frame}

\begin{frame}
    \frametitle{Графическое приложение}
    \textbf{Возможности:}
    \begin{itemize}
        \item Текстовый редактор RBNF-грамматик с синтаксическим разбором
        \item Визуализация синтаксических диаграмм
        \item Интерактивное редактирование: перемещение, выделение объектов
        \item Преобразования: устранение левой рекурсии, левая факторизация
        \item Анализ: проверка LL(1), вычисление FIRST/FOLLOW
        \item Система Undo/Redo
        \item Загрузка/сохранение грамматик
    \end{itemize}
    
    \vspace{0.3cm}
    
    \textbf{Технологии:} Dear ImGui, DirectX 11 (Windows)
\end{frame}

\begin{frame}
  \frametitle{Метрики}
  \begin{center}
  \begin{tabular}{|l|c|c|}
  \hline
  \textbf{Компонент} & \textbf{Файлы} & \textbf{Строк кода} \\
  \hline
  Библиотека libsyngt & 70 & ~7700 \\
  Консольное приложение & 1 & ~400 \\
  Графическое приложение & 1 & ~600 \\
  Модульные тесты & 18 & ~1800 \\
  \hline
  \textbf{Всего} & \textbf{92} & \textbf{~10700} \\
  \hline
  \end{tabular}
  \end{center}
  
  \vspace{0.5cm}
  
  \textbf{Сравнение с оригиналом:}
  \begin{itemize}
      \item Pascal: ~12000 строк (без GUI), 0 тестов, Windows only
      \item C++: ~7700 строк (библиотека), 18 тестовых модулей, кросс-платформенность
  \end{itemize}
\end{frame}

\begin{frame}
  \frametitle{Результаты}
  \begin{itemize}
    \item Успешно портирована система SynGT с Pascal на C++17
    \item Создана модульная архитектура с разделением логики и представления
    \item Реализована библиотека libsyngt для работы с RBNF-грамматиками
    \item Разработаны консольное и графическое приложения
    \item Создан набор из 18 тестовых модулей (100\% прохождения)
    \item Обеспечена кросс-платформенность (Windows, Linux, macOS)
    \item Код доступен в репозитории: \url{https://github.com/stanislavgexrby/vkr}
  \end{itemize}
  
  \vspace{0.3cm}
  
  \textbf{Дальнейшие направления:}
  \begin{itemize}
      \item Расширение поддержки RBNF-конструкций
      \item Экспорт в SVG/PDF
      \item Веб-версия системы
  \end{itemize}
\end{frame}
%\addtocounter{framenumber}{1}
\end{document}